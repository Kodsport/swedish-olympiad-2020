\problemname{Beslutsångest}
Nikolaj bor i en stad som består av ett $N \times M$-rutnät, där raderna är numrerade från $1$ till $N$ och kolumnerna
är numrerade från $1$ till $M$. En cell i rutnätet på rad nummer $r$ och kolumn nummer $c$ brukar betecknas $(r, c)$.
Eftersom Nikolaj inte har något jobb har han varit 
på en massa intervjuer och fått erbjudanden av $K$ stycken företag. Det $i$:te företaget har sitt kontor i cellen 
$(r_i, c_i)$, och inga två företag har kontor på samma cell. Det enda som återstår för Nikolaj är att gå till en
 av dessa celler, så blir han direkt anställd på företaget som har sitt kontor där. För att bestämma vilken cell 
 han ska gå till har Nikolaj mätt varje jobbs \textit{nyttighet}, som är ett heltal mellan $-10^9$ och $10^9$.
 Ett jobbs nyttighet är ett mått på hur mycket nytta Nikolaj skulle göra om han började där. Att sitta hemma och 
 göra ingenting har t.ex. nyttan $0$.
 
 Nu kanske det låter som att Nikolajs val borde vara enkelt, men tyvärr är det långt ifrån sant. Nikolaj tycker nämligen att 
 ju mindre nyttigt ett jobb är, desto roligare är det. Han kan därför inte bestämma sig för om han borde maximera eller
 minimera nyttigheten. Nikolaj kommer starta vid någon cell $(r, c)$, och från början är han inställd på att minimera
 nyttigheten. Varje sekund kommer han gå till antingen cell $(r+1, c)$ eller $(r, c+1)$, och efter att ha gått ett steg
 ändrar han sig och vill istället maximera nyttigheten. Därefter upprepas processen så att han omväxlande vill minimera, maximera, 
 minimera, maximera osv. Detta upprepas tills dess att Nikolaj hamnar på ett av företagens kontor, eller hamnar utanför
 rutnätet. Om Nikolaj missar alla kontor och hamnar utanför rutnätet får han inget jobb och uppnår därmed nyttighet $0$.
 
 När Nikolaj väljer vilken cell han ska gå till gör han det alltid optimalt så att nyttan minimeras/maximeras, 
 och han är medveten om sin ``andra personlighet'' och hur den beter sig.
 Om du vill kan du tänka dig situationen som ett spel där Nikolajs två personligheter spelar mot varandra.
 Om Nikolaj startar vid en cell som är ett kontor blir han direkt anställd där innan han hinner gå någonstans.
 
 Nikolaj har grubblat mycket på hur startpunkten $(r, c)$ påverkar vilket jobb han till slut får. Låt oss säga 
 att nyttigheten Nikolaj uppnår om han startar vid $(r, c)$ är $N(r, c)$. Din uppgift är att räkna ut summan av 
 $N(r, c)$ över alla de $NM$ cellerna i rutnätet. Eftersom talet kan bli ganska stort ska du skriva ut det modulo $998244353$.
 

\section*{Indata}
Den första raden innehåller tre heltal $N$, $M$ och $K$ ($1 \leq N,M \leq 10^9$, $1 \leq K \leq \min(NM, 10^5)$).

De följande $K$ raderna innehåller tre heltal $r_i$, $c_i$ och $v_i$, vilket betyder att det $i$:te företaget har sitt
kontor på cellen $(r_i, c_i)$ och har nyttighet $v_i$ ($1 \leq r_i \leq N$, $1 \leq c_i \leq M$, $-10^9 \leq v_i \leq 10^9$).

\section*{Utdata}
Skriv ut ett heltal, summan av den slutgiltiga nyttigheten över alla de $NM$ cellerna, modulo $998244353$.

\section*{Poängsättning}
Din lösning kommer att testas på en mängd testfallsgrupper.
För att få poäng för en grupp så måste du klara alla testfall i gruppen.

\noindent
\begin{tabular}{| l | l | l |}
\hline
Grupp & Poängvärde & Gränser \\ \hline
$1$   & $24$       & $N,M \leq 1000$. \\ \hline
$2$   & $20$       & $N \leq 5$ \\ \hline
$3$   & $12$       & $N,M \leq 10^5$ \\ \hline
$4$   & $15$       & $K \leq 1000$ \\ \hline
$5$   & $29$       & Inga ytterligare begränsningar. \\ \hline

\end{tabular}
