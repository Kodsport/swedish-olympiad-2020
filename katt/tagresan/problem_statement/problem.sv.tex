\problemname{Tågresan}
En grupp personer ska åka tåg och funderar hur de ska sitta.
$M$ par av personerna är vänner och vill sitta så nära varandra som möjligt.
Tågvagnen de ska sitta i har $N$ rader och är utformad som ett $N \times 4$ rutnät, med 4 platser på varje rad.
Det finns $4N$ personer i gruppen.
Personerna är numrerade $1$ till $4N$.

Varje par av vänner som sitter på platser med euklidiskt avstånd $L = \sqrt{dx^2 + dy^2}$
bidrar med $1/L^2$ till gruppens totala lycka.
Din uppgift är att placera ut personerna så att gruppens lycka blir så stor som möjligt.

\section*{Indata}
Indatan består av $10$ testfall.

Den första raden innehåller talet $T$ ($0 \le T \le 10$), som beskriver numret på testfallet ($0$ för sample).
Den andra raden innehåller talen $N$ och $M$ ($1 \le N \le 25\,000$, $1 \le M \le 100\,000$), antalet rader i tågvagnen samt antalet par av vänner.
De följande $M$ raderna innehåller två heltal $a$ och $b$ ($1 \le a,b \le 4N$, $a \ne b$), nummer för två vänner.

\section*{Utdata}
Skriv ut $N$ rader med 4 heltal på varje.
Varje rad ska innehålla nummer för de fyra personer som sitter på den raden.
Alla personer ska placeras ut någonstans.

\section*{Poängsättning}
Ditt program kommer att testas på $10$ testfall, och poängsättas baserat på hur det står sig relativt \emph{alla andra deltagare}.
Om din inskickning på ett givet testfall får en summa av lycka som är $X$, och den bästa summan någon har uppnått på det testfallet är $Y$,
så blir din poäng på det testfallet $10 \cdot (X / Y)^2$, avrundat till två decimaler.

Din poäng för en inskickning är summan av poängen för testfallen, och din totala poäng är poängen för den bästa inskickningen.
Det går alltså inte att kombinera testfall mellan olika inskickningar, utan du måste skicka in ett enda program som
maximerar din poäng på alla testfall samtidigt.

Under själva tävlingen kommer din poäng sättas relativt en domarlösning;
efter att den slutat kommer alla lösningar att rättas om mot de bästa deltagarlösningarna.
Om din lösning är bättre än den domarna slängt ihop är det därmed möjligt att du tillfälligt får mer än $10$ poäng på ett testfall.
Din (temporära) totalpoäng kommer dock att begränsas till $100$.

Sampletestfallet kommer alltid att ge $0$ poäng.


\section*{Förklaring av exampelfall}
I exempelfallet vill vi placera ut 8 personer i ett $2 \times 4$ rutnät.
Exempellösningen optimerar avståndet för alla vänskapsrelationer utom 1 och 4 som sitter på avstånd $\sqrt 3$ från varandra.
Summan av lycka blir $4 \cdot 1/1 + 1/3 \approx 4.33$.

En bättre lösning kan fås till så att alla par av vänner sitter på avstånd 1.
Om det testfallet have varit ett riktigt testfall och en annan deltagare hade genererat denna lösning
(total lycka $5$) så hade testfallet getts $10 \cdot (4.33 / 5)^2 \approx 7.51$ poäng.

\section*{Testfall}
Här är beskrivningar av de $10$ testfallen som förekommer:

\noindent
\begin{tabular}{| l | l | l |}
\hline
Testfall & Gränser                       & Beskrivning av testfall \\ \hline
$1$       & $N = 10$, $M = 100$           & Alla kanter är genererade slumpmässigt. \\ \hline
$2$       & $N = 100$, $M = 500$          & Alla kanter är genererade slumpmässigt. \\ \hline
$3$       & $N = 100$, $M = 8\,000$       & Alla kanter är genererade slumpmässigt. \\ \hline
$4$       & $N = 8\,000$, $M \le 45\,000$ & $G$ är uppbyggd av grupper av vänner i storlek $1$-$5$ där alla känner alla. \\ \hline
$5$       & $N = 2\,500$, $M = 8\,000$    & $G$ innehåller inga cykler. Det maximala avståndet i $G$ är $2$. \\ \hline
$6$       & $N = 2\,500$, $M = 9\,800$    & $G$ innehåller inga cykler. Det maximala avståndet i $G$ är $2$. \\ \hline
$7$       & $N = 2\,500$, $M = 9\,999$    & $G$ innehåller inga cykler. \\ \hline
$8$       & $N = 25\,000$, $M = 99\,999$  & $G$ innehåller inga cykler. \\ \hline
$9$       & $N = 10\,000$, $M = 100\,000$ & Alla kanter är genererade slumpmässigt. \\ \hline
$10$      & $N = 25\,000$, $M = 100\,000$ & Alla kanter är genererade slumpmässigt. \\ \hline
\end{tabular}

Med slumpmässig menas här \href{https://sv.wikipedia.org/wiki/Likformig_sannolikhetsf\%C3\%B6rdelning}{likformigt slumpmässig}. 
Grafen av vänskapsrelationer kallas $G$ i vissa av beskrivningarna.
