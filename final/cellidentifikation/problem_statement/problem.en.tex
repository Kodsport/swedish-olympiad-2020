\problemname{Identifying Cells}
\noindent
After receiving yet another Wrong Answer verdict, even though your program was guaranteed to be completely correct
this time, you have decided to take a break from competitive programming. You are now studying biology instead, more
specifically, you are interested in the cells of your favorite cactus (their green color reminds you of Accepted verdicts).

In the sample you are looking at, unfortunately, there are all sorts of cells, and it's not entirely easy to
know if you are really looking at your favorites. Different types of cells have different components that they
can be identified by. For example, most cells have Golgi apparatuses, but only plant cells have vacuoles.
Identification is further complicated by the fact that your cheap microscope doesn't always manage to see all the components in a cell.

Your biology book describes the components of $N$ different cell types. There are a total of $K$ possible
components, and no two cell types consist of exactly the same parts.

In order to aid your research, you will create a program that, given the components that you can see in a given cell,
will attempt to identify its type.
Because there are so many cells in your sample, it needs to do this $Q$ times.

\section*{Input}
The first line of the input contains the integers $N$ and $K$ ($1 \leq N \leq 2 \cdot 10^5$, $1 \leq K \leq 21$),
the number of cell types in your book and the number of possible components.

The following $N$ lines each describe a cell type from your book, the first line describing the first one, 
the second line describes the second one and so on. Each line contains a string containing $K$ ones or zeros,
where the $i$:th character is a one if the cell type contains component $i$.

The next line contains a single integer $Q$ ($1 \leq Q \leq 2\cdot 10^5$), the number of queries.

The following $Q$ lines each describe a query. Each line contains a description of the components you can \textit{see}
in the microscope, in the same format as the cells in your book. A one in position $i$ means that you know that the cell
you are looking at has component $i$. However, it may be the case that the cell contains more components that you can't
see with your microscope. Also note that it is possible that two different cells both have a certain component, but
you are only able to see it in one of them.

\section*{Output}
For each query, you should print a line containing the answer.

\begin{itemize}
  \item If the cell type can be decided unambiguously, print the \textit{index} $1 \leq i \leq N$ of the type of the cell.
  \item If there are multiple possibilites, you should print ``\texttt{vet ej}'' (don't know).
  \item If there are no matching cell types, you should print ``\texttt{finns ej}'' (doesn't exist).
\end{itemize}

\section*{Points}
Your solution will be tested on several test case groups.
To get the points for a group, it must pass all the test cases in the group.

\noindent
\begin{tabular}{| l | l | p{12cm} |}
  \hline
  \textbf{Group} & \textbf{Point value} & \textbf{Constraints} \\ \hline
  $1$     & $15$         & $ K \le 8$, $N, Q \leq 10$\\ \hline
  $2$     & $30$         & $ N, Q \leq 1000$\\ \hline
  $3$     & $55$         & No further constraints.\\ \hline
\end{tabular}

