\problemname{Gruppindelning}
Matteläraren Maria tänker dela in sina elever i två grupper under nästa lektion.
Därför ställs hon nu inför ett vanligt problem: hur delar man in elever i två grupper på ett vettigt sätt?
Det finns $n$ elever i klassen, och klassrummet har $n$ stolar, numrerade från $1$ till $n$.
När en elev anländer till klassrummet så sätter den sig alltid på den lediga stolen som är längst till vänster.
Så om totalt $k$ personer dyker upp en viss dag så kommer de alltid att sitta på stolarna $1,2,...,k$.

För att dela in eleverna i två grupper har Maria en viss strategi som hon vill använda.
Innan lektionen börjar väljer hon en sekvens $s$ av $n$ st ettor och tvåor.
När lektionen har börjat så går hon till varje elev och låter eleven som sitter på stol $i$ hamna i grupp $s_i$.
För att gruppindelningen ska bli vettig så finns det två krav som $s$ måste uppfylla:

\begin{enumerate}
  \item Maria vet inte hur många elever som kommer, men oavsett hur många som dyker upp så ska skillnaden i storlek mellan de två grupperna vara högst $1$.

  \item Det finns $m$ par av stolar som har samma färg. Om det sitter elever på ett sådant par av stolar så måste de hamna i samma grupp.
\end{enumerate}

Din uppgift är att, givet $n,m$ och de $m$ paren av stolar, hitta den sekvens $s$ som kommer först i alfabetisk ordning och uppfyller kraven ovan.
Om det inte finns något giltigt $s$ ska ditt program skriva ut $-1$.

Med alfabetisk ordning menas att man jämför två sekvenser genom att primärt kolla på det första tecknet, om det är samma kollar man på det andra, o.s.v.
T.ex. kommer sekvensen \texttt{1122} före \texttt{1211}, men efter \texttt{1112}.

\section*{Indata}
Den första raden innehåller två heltal $1\leq n \leq 10^5$, antalet stolar i klassrummet och $0\leq m \leq n/2$, antalet stolpar med samma färg.
Därefter följer $m$ rader som vardera består av två heltal, de par av stolar som har samma färg (1-indexerat).
Varje stol har samma färg som högst en annan.

\section*{Utdata}
Skriv ut en sekvens med $n$ ettor eller tvåor (utan mellanslag), alternativt $-1$ om det saknas en giltig lösning.

\section*{Poängsättning}
Din lösning kommer att testas på en mängd testfallsgrupper.
För att få poäng för en grupp så måste du klara alla testfall i gruppen.

\noindent
\begin{tabular}{| l | l | l |}
\hline
Grupp & Poängvärde & Gränser \\ \hline
1     & 12         & $n \leq 15$\\ \hline
2     & 23         & $n \leq 40$ \\ \hline
3     & 25         & Stol $i$ kan bara ha samma färg som stol $i\pm 1$ \\ \hline
4     & 40         & Inga ytterligare begränsningar \\ \hline
\end{tabular}

\section*{Förklaring av exempelfall 1}
Om det bara kommer två personer så måste de vara i olika grupper.
Alltså måste stol $1$ och stol $2$ vara i olika grupp, men vi vet också att stol $3$
måste vara i samma grupp som stol $2$. Om det kommer fyra personer måste därför stol $4$
vara i samma grupp som stol $1$ för att grupperna ska bli lika stora. Men stol $4$ måste
vara i samma grupp som stol $5$. Om vi fortsätter på samma sätt får vi att stol $1, 4$
och $5$ måste vara i samma grupp, medan stol $2, 3, 6$ och $7$ är i den andra gruppen.
Det finns alltså två lösningar: $1221122$ och $2112211$. Den alfabetiskt minsta av
dessa är $1221122$.
