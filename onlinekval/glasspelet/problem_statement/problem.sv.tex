\problemname{Glasspelet}
Kirderf och Slin har kommit tidigt till en stor fest där glass serveras till alla.
Just nu är de ensamma i festlokalen där alla $N$ glassar är uppställda på en lång rad på ett bord.
Den finns $K$ olika slags glassar, där varje sort representeras av ett heltal mellan $0$ och $K-1$.

Slin och Kirderf är väldigt sugna på glass och tänker därför spela ett spel där de äter av glassarna.
De gör vartannat drag, där ett drag går ut på att äta exakt en glass.
När som helst kan festens organisatörer komma in i rummet, och då får det inte vara för
uppenbart att Kirderf och Slin har ätit av glassarna. Därför kan de endast välja att
äta glassen längst till vänster eller längst till höger, så att det inte blir några mellanrum
i raden med glassar. Dessutom måste det alltid finnas minst en av varje glassort kvar,
annars är det ju uppenbart att någon har ätit. Den som inte kan göra något drag förlorar.

Kirderf vill absolut inte förlora mot Slin, därför har han bett dig skriva ett program som avgör vem som har en vinnande
strategi vid olika tillstånd av spelet. Du kommer få $Q$ st frågor, där varje fråga är ett intervall $[L,R]$, och 
ditt program ska avgöra vem som har en vinnande strategi om bara glass nummer $L, L+1, \dots, R$ är kvar (glassarna
är numrerade från $1$ till $N$ från vänster till höger).

Att en spelare har en vinnande strategi innebär att den spelaren alltid kan se till att vinna spelet,
oavsett vilka drag motståndaren gör. Det går att bevisa att någon av Kirderf eller Slin alltid har en vinnande strategi.

\section*{Indata}
Den första raden innehåller tre heltal $N$, $K$, $Q$ ($1 \le K \le N \le 2 \times 10^5$, $1 \le Q \le 10^5$).
Den andra raden innehåller $N$ heltal, alla mellan $0$ och $K-1$.
Dessa beskriver hur raden med glassar ser ut. Varje glassort förekommer minst en gång.
De följande $Q$ raderna innehåller två heltal $L_i$ och $R_i$ ($1 \le L_i \le R_i \le N$).


\section*{Utdata}
Skriv ut $Q$ rader med ett heltal för varje fråga. Om den spelare som börjar vid tillståndet $[L_i, R_i]$ har en 
vinnande strategi, skriv ut $1$. Om den andra spelaren har en vinnande strategi, skriv ut $2$. Om tillståndet inte 
är giltigt, dvs om varje glassort inte finns i intervallet, skriv ut $0$.

\section*{Poängsättning}
Din lösning kommer att testas på en mängd testfallsgrupper.
För att få poäng för en grupp så måste du klara alla testfall i gruppen.

\noindent
\begin{tabular}{| l | l | p{12cm} |}
  \hline
  Grupp & Poängvärde & Gränser \\ \hline
  $1$   & $9$        & $N,Q \le 15$ \\ \hline
  $2$   & $11$       & $N,Q \le 100$ \\ \hline
  $3$   & $17$       & $N,Q \le 1000$ \\ \hline
  $4$   & $5$        & $K=1$ \\ \hline
  $5$   & $7$        & $K=2$ \\ \hline
  $6$   & $11$       & $R_i - L_i + 1 \le K+1$ för alla $1 \le i \le Q$ \\ \hline
  $7$   & $23$       & $Q=1$, $L_1=1, R_1=N$ \\ \hline
  $8$   & $17$       & Inga ytterligare begränsningar. \\ \hline
\end{tabular}


\section*{Förklaring av exempelfall 1}
I tillståndet $[1,5]$ (dvs alla $5$ glassar är kvar) har den andra spelaren en vinnande strategi. 
Oavsett vad den första spelaren gör har den andra spelaren ett möjligt drag, och efter det finns bara $3$ glassar kvar
vilket innebär att den första spelaren förlorar.

I tillståndet $[1,4]$ har den första spelaren en vinnande strategi, nämligen att ta bort glass nummer $1$.

Tillståndet $[1,3]$ är inte giltigt eftersom ingen glass av sort $2$ finns bland de tre första.

