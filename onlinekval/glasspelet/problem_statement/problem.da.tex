\problemname{Islegen}
Frederik og Nils er kommet tidligt til en fest, hvor der serveres is til alle.
Lige nu er de alene i festlokalet, hvor alle $N$ is står på én lang række på et bord.
Der er $K$ forskellige slags is, hver slags er repræsenteret af et heltal mellem $0$ og $K-1$.

Nils og Frederik er meget lækkersultne, og de vil derfor spille et spil, som går ud på at spise is.
De skiftes til at trække, og et træk består af at spise netop én is.
Festudvalget kan komme ind i lokalet lige hvornår det skal være, og det bør ikke være for åbenlyst, at Nils og Frederik har smugspist af isene.
Derfor kan de kun spise is enten fra længst til venstre eller længst til højre, så der ikke opstår nogle mellemrum i isrækken.
Derudover skal der altid være mindst en af hver slags is tilbage, ellers er det jo tydeligt, at nogen har spist is.
Den spiller, der ikke kan gøre et lovligt træk, har tabt.

Frederik vil absolut ikke tabe mod Nils, derfor har han bedt dig om at skrive et rogram til at afgøre, hvem der har den vindende strategi ved forskellige tilstande i spillet.
Du vil få $Q$ spørgsmål, hvor hvert spørgsmål er et interval $[L,R]$, og dit program skal afgøre, hvem der har en vindende strategi, når kun isene nummer $L$, $L+1$, $\dots$, $R$ er tilbage (isene er nummererede fra $1$ til $N$ fra venstre til højre).

At en spiller har en vindende strategi betyder, at den spiller altid kan vinde spillet uanset modstanderens træk.
Man kan vise, at enten Frederik eller Nils altid har en vindende strategi.

\section*{Indlæsning}
Første linje består af tre heltal $N$, $K$, $Q$ ($1 \le K \le N \le 2 \times 10^5$, $1 \le Q \le 10^5$).
Anden linje består af $N$ heltal, alle mellem $0$ og $K-1$.
Disse beskriver rækken af is.
Hver slags is forekommer mindst én gang.
De følgende $Q$ linjer består af to heltal $L_i$ og $R_i$ ($1 \le L_i \le R_i \le N$).

\section*{Udskrift}
Udskriv $Q$ linjer med et heltal for hvert spørgsmål.
Hvis den spiller, hvis tur det er i tilstanden $[L_i, R_i]$, har en vindende strategi, udskriv~$1$.
Hvis den anden spiller har en vindende strategi, udskriv~$2$.
Hvis tilstanden ikke er gyldig, dvs.\ hvis ikke hver slags is forekommer i intervallet, udskriv~$0$.

\section*{Pointgivning}
Din løsning køres på otte forskellige grupper af test.
For at få point for en gruppe, skal du klare samtlige test i gruppen.

\noindent
\begin{tabular}{| l | l | l |}
\hline
Gruppe & Pointi    & Begrænsninger \\ \hline
$1$     & $9$          &  $N,Q \le 15$ \\ \hline
$2$     & $11$         &  $N,Q \le 100$  \\ \hline
$3$     & $17$         &  $N,Q \le 1000$  \\ \hline
$4$     & $5$          &  $K=1$  \\ \hline
$5$     & $7$          &  $K=2$  \\ \hline
$6$     & $11$         &  $R_i - L_i + 1 \le K+1$ for alle $1 \le i \le Q$. \\ \hline
$7$     & $23$         &  $Q=1$, $L_1=1, R_1=N$ \\ \hline
$8$     & $17$         &  Ingen yderligere begrænsninger. \\ \hline
\end{tabular}

\section*{Forklaring af eksemplet}
I tilstanden $[1,5]$ (dvs.\ alle $5$ is er tilbage) har anden spiller en vindende strategi.
Uanset hvad første spiller måtte vælge, findes der et muligt træk for anden spiller, hvorefter der bare findes $3$ is tilbage -- dermed har første spiller tabt.

I tilstanden $[1,4]$ har første spiller en vindende strategi, nemlig at fjerne is nummer~1.

Tilstanden $[1,3]$ er ugyldig, idet der ikke findes nogen is af slagsen »$2$« blandt de første tre is.
