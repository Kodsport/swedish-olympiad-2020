\problemname{Öar}
2020 års International Olympiad in Informatics (IOI) kommer att avgöras i Singapore, ett till ytan litet land som består av massor av öar.
På en av utflykterna på IOI ska de $N$ deltagarna besöka dessa öar.
Men deltagarna går och tänker på hur de ska implementera Fibonacci-heapar, så en efter en går vilse och hittar inte tillbaka.

På första ön försvinner en deltagare, på andra ön försvinner ytterligare en deltagare.
Om $a_i$ är antalet deltagare som försvinner på ö $i$, så försvinner $a_{k-2} + a_{k-1}$ från ö $k$ (eller antalet kvarvarande deltagare, om det är värre kvar).

\begin{figure}[h]
  \centering
      \includegraphics[width=1.0\textwidth]{oarfig}
      \caption{Figuren visar situationen i det första exemplet när den sista (tolfte) deltagaren försvunnit på ö nummer 5.}
\end{figure}

På vilken ö försvinner den sista deltagaren?

\section*{Indata}
Den första raden innehåller ett heltal $1\le N \le 10\,000$, antalet deltagare.

\section*{Utdata}
Ett heltal $A$, numret på ön där den $N$:te deltagaren försvinner.

\section*{Poängsättning}
Din lösning kommer att testas på en mängd testfallsgrupper.
För att få poäng för en grupp så måste du klara alla testfall i gruppen.

\noindent
\begin{tabular}{| l | l | l |}
  \hline
  Fall & Poängvärde & Gränser \\ \hline
  $1$    & $40$        &  Exakt $a_k$ deltagare försvinner från den sista ön. \\ \hline 
  $2$    & $60$        &  Inga ytterligare begränsningar. \\ \hline
\end{tabular}
