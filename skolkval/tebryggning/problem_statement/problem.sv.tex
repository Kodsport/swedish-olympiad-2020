\problemname{Tebryggning}
Egon ska brygga massor av te till $N$ programmeringsolympiadsdeltagare.
Han har $K$ påsar te, alla av olika sorter.
Påse $i$ har te för $x_i$ personer.
Det är garanterat att påsarna sammanlagt räcker till minst $N$ personer.

Egon tänker använda bryggkannor som har plats för te till maximalt 10 personer.
Eftersom påsarna är av olika sort
går det inte att blanda flera påsar i samma kanna.
Dock kan samma påse användas till flera kannor.
Hur många kannor måste Egon använda?

\section*{Indata}
På den första raden står två heltal $1 \le K \le 10$ och $1 \le N \le 100$ 
 -- antalet tepåsar Egon har och antalet programmeringsolympiadsdeltagare. 
På den andra raden står $K$ heltal $1 \le x_1, x_2, \dots, x_K \le 100$,
antal personer som varje påse räcker till.

\section*{Utdata}
Programmet ska skriva ut ett heltal: det minsta antalet tekannor Egon måste använda. 

\section*{Poängsättning}
För testfall värda $20$ poäng gäller att $K=1$. \\
För de resterande testfallen (värda $80$ poäng) gäller att $1\leq N\leq 100$ och $1\leq K\leq 10$.

\section*{Poängsättning}
Din lösning kommer att testas på fem olika testfall.

\noindent
\begin{tabular}{| l | l | l |}
  \hline
  Fall & Poängvärde & Gränser \\ \hline
  $1$    & $20$        &  $K = 1$\\ \hline 
  $2$    & $20$        &  Inga ytterligare begränsningar. \\ \hline
  $3$    & $20$        &  Inga ytterligare begränsningar. \\ \hline
  $4$    & $20$        &  Inga ytterligare begränsningar. \\ \hline
  $5$    & $20$        &  Inga ytterligare begränsningar. \\ \hline
\end{tabular}


\section*{Förklaring av exempel}
I exempel 1 väljer Egon att brygga två kannor med första tepåsen 
och två kannor med tredje tepåsen. Det ger $20+17$ koppar te, vilket
räcker till de 36 deltagarna.

I exempel 2 är det optimala att brygga sex kannor med första tepåsen,
tre kannor med tredje tepåsen och två med den fjärde tepåsen.
Det ger $54+30+16$ koppar te, vilket räcker till de 100  deltagarna.
