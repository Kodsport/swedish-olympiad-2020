\problemname{Tebryggning}
Egon ska brygga massor av te till $N$ programmeringsolympiadsdeltagare.
Han har $K$ påsar te, alla av olika sorter.
Påse $i$ har te för $x_i$ personer.
Det är garanterat att påsarna sammanlagt räcker till minst $N$ personer.

Egon tänker använda bryggkannor som har plats för te till maximalt 10 personer.
Eftersom påsarna är av olika sort
går det inte att blanda flera påsar i samma kanna.
Dock kan samma påse användas till flera kannor.
Hur många kannor måste Egon använda?

\section*{Indata}
Den första raden innehåller två heltal $K$ och $N$ ($1 \le K \le 10$, $1 \le N \le 100$), antalet tepåsar Egon har och antalet programmeringsolympiadsdeltagare. 

Den andra raden innehåller de $K$ heltalen $x_i$ ($1 \le x_i \le 100$), antalet personer som varje påse räcker till.

Det är garanterat att tepåsarna alltid räcker till $N$ personer.

\section*{Utdata}
Programmet ska skriva ut ett heltal: det minsta antalet tekannor Egon måste använda. 

\section*{Poängsättning}
Din lösning kommer att testas på en mängd testfallsgrupper.
För att få poäng för en grupp så måste du klara alla testfall i gruppen.

\noindent
\begin{tabular}{| l | l | l |}
  \hline
  Fall & Poängvärde & Gränser \\ \hline
  $1$    & $20$        &  $K = 1$\\ \hline 
  $2$    & $80$        &  Inga ytterligare begränsningar. \\ \hline
\end{tabular}


\section*{Förklaring av exempel}
I det första exemplet väljer Egon att brygga två kannor med första tepåsen och två kannor med tredje tepåsen.
Det ger $20+17$ koppar te, vilket
räcker till de $36$ deltagarna.

I det andra exemplet är det optimala att brygga sex kannor med första tepåsen,
tre kannor med tredje tepåsen och två med den fjärde tepåsen.
Det ger $54+30+16$ koppar te, vilket räcker till de $100$ deltagarna.
